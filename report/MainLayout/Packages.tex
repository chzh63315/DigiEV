%%% Packages %%%
\usepackage{multirow}
\usepackage[T1]{fontenc}
\usepackage{ragged2e}
\usepackage{listings}
%%\usepackage[latin1, utf8]{inputenc}
\usepackage{lmodern} 
\usepackage[a4paper,right=2.75cm, left=2.75cm, top=3.cm, bottom=4.cm]{geometry}
\usepackage[english]{babel} % You can change to english if needed
\usepackage{afterpage}
\usepackage[colorlinks=true, urlcolor=blue, citecolor=blue, linkcolor=black]{hyperref}
% \setSingleSpace{1.1} % Space between lines of the text
\usepackage{calc, blindtext}
\usepackage[table]{xcolor}
\usepackage{nicematrix, booktabs}
\usepackage{graphicx,soul}
\usepackage{enumitem,lipsum}
\usepackage{float} % For the [H] option in tables
\usepackage{minted}
\usepackage{adjustbox}
\usepackage{mathtools,amsfonts,amsthm,amssymb,amsmath}
\usepackage{fancyvrb}  % pour afficher proprement du code LaTeX
\usepackage{array}
\usepackage{siunitx}
\sisetup{detect-all} % note: just one option still needs to be specified
\usepackage{tikz}
\definecolor{doc}{RGB}{0,60,110}
\usepackage{titletoc}
\setcounter{tocdepth}{3}
\usepackage{pdfpages}
\usepackage{titletoc}
\usepackage[figurename=Fig., tablename=Tab.]{caption}
\usepackage{subcaption}
\usepackage[acronym]{glossaries}
\makeglossaries
\usepackage{nameref}
\renewcommand{\thesection}{\arabic{section}}

%%% Colors %%%
\definecolor{CentraleBlue}{RGB}{11,15,157}   % Blue of the Centrale Med Logo
\definecolor{codebg}{rgb}{0.95,0.95,1}       % Light blue background for the code box

%%% Custom Chapter Style %%%
\makechapterstyle{HansenColor}{
  \chapterstyle{Hansen}               
  \renewcommand{\printchaptername}{}  % Remove "Chapter" word
  \renewcommand{\printchapternum}{}   % Remove chapter number
  \renewcommand{\afterchapternum}{}   % Remove spacing left by the number

  % Set the chapter title in blue (customize color as needed)
  \renewcommand*{\printchaptertitle}[1]{\chaptitlefont\color{CentraleBlue}##1} 
}

%%% Maths %%% 

% Theorems, definitions, propositions etc.
\theoremstyle{plain}
\newtheorem{theorem}{Théorème}[section]
\newtheorem{proposition}[theorem]{Proposition}
\newtheorem{lemma}[theorem]{Lemme}
\newtheorem{corollary}[theorem]{Corollaire}

\theoremstyle{definition}
\newtheorem{definition}[theorem]{Définition}
\newtheorem{remark}[theorem]{Remarque}

% You can define your own commands
\newcommand{\R}{\mathbf{R}}
\newcommand{\N}{\mathbf{N}}
\newcommand{\Q}{\mathbf{Q}}
\newcommand{\Z}{\mathbf{Z}}
\newcommand{\CC}{\mathbf{C}}
\newcommand{\E}{\mathbf{E}}
\newcommand{\D}{\ensuremath{\mathrm{d}}}
\newcommand{\Cov}{\mathrm{Cov}}
\newcommand{\Var}{\mathrm{Var}}
