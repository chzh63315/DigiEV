\chapter{Introduction}\label{chap1}

The global transition toward electric mobility is regarded as a cornerstone of 
efforts to decarbonize the transport sector and meet international climate 
targets. According to the International Energy Agency, electric car sales 
reached nearly 14 million in 2023, and are projected to exceed 17 million 
in 2024, representing more than one-fifth of global car sales~\cite{IEA2024}. 
However, this rapid diffusion of electric vehicles (EVs) poses systemic 
challenges that extend beyond vehicle manufacturing. The widespread adoption 
of EVs crucially depends on the development of robust charging infrastructure 
and the integration of energy systems capable of supporting increasing loads 
while ensuring reliability and user acceptance~\cite{Metais2022}. 

Charging infrastructure represents both a key enabler and a potential bottleneck 
in the electrification of mobility. Without adequate charging coverage, 
prospective EV users may face range anxiety, undermining confidence in the 
technology and slowing adoption~\cite{Metais2022}. At the same time, 
deploying charging networks entails high capital costs and introduces 
technical challenges related to grid integration, station placement, and 
operational optimization. France, for instance, has set ambitious goals of 
400,000 publicly accessible charging points by 2030, yet as of 2023 fewer 
than 100,000 had been deployed, revealing the scale of investment and 
coordination required~\cite{Autorite2023}. This “chicken-and-egg” 
problem—insufficient infrastructure limiting EV adoption, while weak demand 
discourages infrastructure investment—highlights the importance of 
strategically planned deployment~\cite{Bernard2021}.

To address these challenges, advanced modeling and simulation tools have been 
proposed to guide infrastructure planning and management. Digital twin 
technology, in particular, offers a powerful approach to replicate complex 
physical systems in a virtual environment, enabling real-time monitoring, 
scenario evaluation, and predictive analytics. Recent studies highlight the 
potential of digital twins to support the integration of charging infrastructure 
with smart grids, optimize location and capacity decisions, and enhance 
decision-making under uncertainty~\cite{Charette2023, Metais2022}. 


This report builds upon this context by presenting the work conducted during 
a six-month internship at the Laboratoire Ville Mobilité Transport (LVMT), 
École nationale des ponts et chaussées (ENPC). The project focused on the 
design and development of a digital twin system for simulating the energy 
impact of EV charging infrastructure, employing the FIWARE framework and 
NGSI-LD standards to ensure interoperability. By coupling backend databases 
(MongoDB, CrateDB) with real-time APIs, the system provides a scalable 
platform for monitoring charging operations and assessing deployment 
strategies. The work contributes to ongoing research in sustainable mobility 
by demonstrating how digital twins can support the optimization of EV 
charging infrastructure within smart city ecosystems.

This report is organized into seven main chapters, each addressing specific aspects 
of the internship work and its broader context:

Chapter 2 (\textbf{Context}) presents the institutional framework of the internship, 
including the presentation of ENPC, LVMT, E4C, and Saclay campus, along with supervisor 
introductions and the background context of energy and climate research, EV charging 
infrastructure, digital twin technologies, and semantic data modeling.

Chapter 3 (\textbf{Literature Review}) examines existing research and technologies, 
including FIWARE, BRICK, Web of Things (WoT), and related digital twin implementations. 
This chapter identifies current challenges and positions our contribution within the 
existing body of knowledge.

Chapter 4 (\textbf{Methods}) details the system architecture, semantic data modeling 
approach using NGSI-LD, backend database implementation, API development, and frontend 
visualization components. It concludes with an analysis of contributions, difficulties 
encountered, and deviations from initial specifications.

Chapter 5 (\textbf{Case Studies and Experiments}) presents the definition and 
initialization of three experimental scenarios, results obtained using both real 
and synthetic data, and validation of the digital twin system performance.

Chapter 6 (\textbf{Management Analysis}) addresses social and environmental 
responsibility aspects of the project and discusses innovation management and digital 
tools for decision-making in resear


