\chapter{Quelques commandes utiles}\label{chap2}

\section{Inclure des images}

\begin{figure}[ht!]
    \centering
    \includegraphics[width=0.7\textwidth]{example-image-a}
    \caption{\textbf{Titre.} \lipsum[1][1-5]}
    \label{fig:image1}
\end{figure}

\begin{figure}[ht!]
    \centering
    \includegraphics[width=0.35\textwidth]{example-image-a}
    \includegraphics[width=0.35\textwidth]{example-image-b}
    \caption{\textbf{Titre.} \lipsum[1][1-5]}
    \label{fig:image2}
\end{figure}



\section{Listes et tableaux}

\textbf{Insérer une liste~:}
\begin{itemize}
    \item Premier niveau
    \begin{itemize}
        \item[(i)] Deuxième niveau
        \item[(ii)] Un autre élément au deuxième niveau
    \end{itemize}
    \item Un autre élément au premier niveau
        \begin{itemize}
        \item[(a)] Deuxième niveau
        \item[(b)] Un autre élément au deuxième niveau
    \end{itemize}
\end{itemize}
\bigskip

\textbf{Insérer un tableau simple~:} 
\begin{table}[H]
    \centering
    \begin{tabular}{|c|c|c|c|c|c|c|c|c|c|}
        \hline
        A & B & C & D & E & F & G & H & I & \dots \\
        \hline
        1 & 2 & 3 & 4 & 5 & 6 & 7 & 8 & 9 & \dots\\
        10 & 11 & 12 & 13 & 14 & 15 & 16 & 17 & 18 & \dots \\
        \hline
    \end{tabular}
    \caption{\textbf{Titre.} \lipsum[1][1-3]}
    \label{tab:table-label}
\end{table}

\textbf{Autre style de tableau~:} 
\begin{table}[H]
    \centering
    \begin{tabular}{c c c c c}
        \hline
        \textbf{Col1} & \textbf{Col2} & \textbf{Col3} & \textbf{Col4} & \textbf{Col5} \\
        \hline
        1 & 2 & 3 & 4 & 5 \\
        1 & 2 & 3 & 4 & 5 \\
        \hline
    \end{tabular}
    \caption{\textbf{Titre.} \lipsum[1][1-3]}
    \label{tab:table-label}
\end{table}


\section{Insertion de code Python}
\textbf{Insérer du code en \textit{inline}~:} \texttt{print("Hello, World!")}.\\


\textbf{Insérer du code en \textit{inline} avec coloration syntaxique~:} \mintinline{python}{print("Hello, World!")}.\\

\textbf{Insérer du code en bloc avec coloration syntaxique~:} 
\begin{minted}[bgcolor=codebg,fontsize=\small,frame=lines,linenos]{python}
def factorial(n):
    if n == 0:
        return 1
    else:
        return n * factorial(n - 1)

# Example usage
print(factorial(5))  # Output: 120
\end{minted}



\section{Expressions mathématiques}
\subsection{Théorèmes, propositions, définitions, lemmes, demonstrations...}

\begin{theorem}
    \lipsum[1][1-4]
\end{theorem}

\begin{proof}
    \lipsum[1][1-4]
\end{proof}

\begin{proposition}
    \lipsum[1][1-4]
\end{proposition}

\begin{definition}
    \lipsum[1][1-4]
\end{definition}

\begin{remark}
    \lipsum[1][1-4]
\end{remark}

\begin{lemma}
    \lipsum[1][1-4]
\end{lemma}


\subsection{Equations et calculs sur plusieurs lignes}

\subsubsection*{Une simple équation}
\begin{equation}
    e = mc^2
\end{equation}

\subsubsection*{Une équation sur plusieurs lignes}
\begin{equation}
    \begin{split}
        \mathbb{E}(aX + Y) &= \mathbb{E}(aX) + Y\\
                           &= a\mathbb{E}(X) + Y
    \end{split}
\end{equation}

\subsubsection*{Une équation avec plusieurs cas}
\begin{equation}
    u_n =
    \begin{cases}
        1 \text{ if } n\equiv0 \mod 2\\
        0 \text{ if } n\equiv1 \mod 2 \\
    \end{cases}
\end{equation}

\subsubsection*{Insérer une série de calculs}
\begin{align*}
    x &= 0.999\ldots \\
    10x &= 9.999\ldots \\
    10x - x &= 9.999\ldots - 0.999\ldots \\
    9x &= 9 \\
    x &= 1 \\
    0.999\ldots &= 1
\end{align*}

\subsection{Utilisation du glossaire}

\newglossaryentry{esperance}{
    name=espérance,
    description={Valeur moyenne théorique d'une variable aléatoire}
}


Référence au glossaire : \gls{esperance}.  




