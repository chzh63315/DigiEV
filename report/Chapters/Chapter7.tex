\chapter{Conclusion}\label{chap7}
\section{Conclusion of the Internship}
This internship provided me with the opportunity to design and implement a digital twin for electric vehicle (EV) charging infrastructure using the FIWARE framework.  
Starting from data modeling with NGSI-LD, I developed an interoperable representation of charging stations, charging points, electric vehicles, and sessions.  
I integrated real datasets collected from the E4C charging stations (2023–2025) and complemented them with synthetic scenarios to test scalability and flexibility.  
The backend architecture, based on MongoDB and CrateDB, successfully supported both real-time context management and historical analysis, while QuantumLeap ensured temporal data persistence.  
Finally, Grafana dashboards provided stakeholders with interactive visualization tools for monitoring energy consumption and infrastructure usage.  

Overall, the project demonstrated the feasibility of combining standardized semantic models with real-world data to create a robust and scalable digital twin.  
This experience also gave me hands-on expertise in both data engineering and IoT system integration, bridging theory and practice.

\section{Future Work}
Although the system is functional, several improvements can be considered for future development:
\begin{itemize}
    \item \textbf{Scalability testing:} Extend performance tests to larger datasets (tens of thousands of sessions) and multiple charging stations to validate system robustness.
    \item \textbf{Advanced analytics:} Incorporate predictive models (e.g., demand forecasting, anomaly detection, and optimal load balancing) to enhance decision-making.
    \item \textbf{Integration with external data and IoT devices:} Extend the system by combining charging data with external data streams such as renewable energy production, weather forecasts, and dynamic electricity pricing. In parallel, integrate IoT devices (e.g., smart meters, charging sensors, and on-site monitoring equipment) to capture real-time operational parameters. This would enable advanced demand-response strategies and provide a more holistic view of the interaction between vehicles, infrastructure, and the energy grid.
    \item \textbf{Enhanced front-end:} Develop a more user-friendly web interface, potentially with dynamic maps and real-time alerts, to complement Grafana dashboards.
    \item \textbf{Standardization efforts:} Contribute to ongoing FIWARE/NGSI-LD data model standardization, especially for EV infrastructure.
\end{itemize}

\section{Career Impact and Personal Development}

This internship significantly contributed to my personal and professional development.  
By aligning with competencies identified in the RNCP framework for the role of \textbf{iot / IoT Architect (Architecte IoT)}, I was able to develop key skills across several domains :contentReference[oaicite:0]{index=0}.

\subsection*{Developed Competencies}
\begin{itemize}
  \item \textbf{Architectural Design and Analysis:} I refined my ability to analyze user requirements and design scalable, secure IoT architectures—from sensor interfaces to data persistence layers—according to RNCP expectations :contentReference[oaicite:1]{index=1}.
  \item \textbf{System Integration and Data Flow Management:} The integration of NGSI-LD data models, Orion Context Broker, MongoDB, CrateDB, QuantumLeap, and Grafana reflects the end-to-end data flow design capabilities expected of an IoT architect :contentReference[oaicite:2]{index=2}.
  \item \textbf{Technical Leadership and Project Execution:} Managing the entire pipeline—data collection, modeling, backend implementation, scripting, simulation experiments—strengthened my ability to coordinate multidisciplinary technical tasks and deliver functioning systems :contentReference[oaicite:3]{index=3}.
  \item \textbf{Innovation and Continuous Learning:} Addressing undocumented FIWARE components, adapting NGSI-v2 tools to NGSI-LD, and refining simulation strategies demonstrated a proactive approach to technological innovation and ongoing knowledge updating :contentReference[oaicite:4]{index=4}.
  \item \textbf{Communication and Documentation:} Producing structured documentation of the system architecture, experimental design, and performance results enhanced my capability to clearly communicate technical concepts to a diverse audience :contentReference[oaicite:5]{index=5}.
\end{itemize}

\subsection*{Impact on My Career Trajectory}
This internship reinforced my interest in research at the crossroads of digital twins, IoT platforms, and sustainable mobility systems.  
Given this deepened motivation and strengthened expertise, I intend to pursue a \textbf{PhD}, focusing on advanced topics such as :
\begin{itemize}
  \item Semantic modeling and standardization for EV infrastructure,
  \item Predictive and optimization methods for smart charging under variable grid conditions,
  \item Scalable architecting of IoT systems in energy-aware ecosystems.
\end{itemize}

In summary, this experience not only consolidated critical technical and methodological competencies aligned with RNCP expectations for IoT professionals, but also clarified and propelled my long-term academic and professional ambitions.
