\chapter{Analysis management}\label{chap6}

\section{Social and Environmental Responsibility (RSE)}

\subsection*{Concepts and Keywords}
Corporate Social and Environmental Responsibility (RSE) refers to the responsibility of organizations for the impacts of their activities on society and the environment.  
Key concepts associated with RSE include: \textit{sustainability, climate action, social equity, human rights, community development, and ethical governance}.  
In the context of electric vehicle (EV) infrastructure, the most relevant priorities are:
\begin{itemize}
    \item \textbf{Environmental sustainability:} reducing greenhouse gas emissions and improving energy efficiency through electrification,
    \item \textbf{Communities and local development:} ensuring accessibility of charging infrastructure for both urban and rural areas,
    \item \textbf{Social equity:} guaranteeing equal opportunities for all users to benefit from the EV transition without discrimination.
\end{itemize}

\subsection*{RSE and Economic Development}
Far from being a constraint, RSE is a driver of sustainable economic growth. Integrating RSE strategies contributes to:
\begin{itemize}
    \item Reducing long-term costs (environmental damage, public health issues),
    \item Opening new markets, especially in clean technologies and green mobility,
    \item Strengthening competitiveness through alignment with European regulations such as the \textbf{EU Green Deal} and \textbf{Fit for 55} package \cite{europeancommission2021greendeal}.
\end{itemize}

For France, RSE is embedded in industrial and environmental policies. The \textit{Stratégie Nationale Bas-Carbone} emphasizes the electrification of transport and the development of circular economy practices, including battery recycling and second-life applications.

\subsection*{Relevance to the Internship Context}
During my internship, RSE principles directly influenced the development of the EV charging digital twin:
\begin{itemize}
    \item \textbf{Energy efficiency:} by modeling charging sessions, it was possible to evaluate energy demand and propose optimizations for reducing peak loads,
    \item \textbf{Integration with renewable energy:} simulation scenarios considered alignment of charging times with solar and off-peak availability,
    \item \textbf{Social impact:} data modeling highlighted the importance of equitable distribution of charging points across regions, preventing disparities in mobility access.
\end{itemize}

Thus, the project not only had a technological dimension but also addressed social and environmental challenges aligned with RSE objectives.

\subsection*{Conclusion and Recommendations}
RSE and economic development are complementary: sustainability concerns set the direction, while innovation provides the tools to achieve these goals.  
For companies and research laboratories, adopting RSE strategies in EV infrastructure implies:
\begin{itemize}
    \item Supporting fair access to charging infrastructure,
    \item Integrating renewable energy into charging networks,
    \item Enhancing transparency and stakeholder engagement through digital twins.
\end{itemize}
In the long term, this alignment strengthens both competitiveness and societal value creation.




