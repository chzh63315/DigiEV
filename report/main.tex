%% This layout has been constructed from the combination of several packages, layout found on internet and from my own construction. 
%% It is done to match the layout of Centrale Méditerranée but can be adapted for other university
%% Author: Vincent G. 2025 (Promo Entrante 2022)


\documentclass[11pt]{memoir}

% Front page (Replace with your own information)
\newcommand{\prenomNOM}{ZHAO Chao} % prénom NOM
\newcommand{\dateDebutStage}{01/04/2025} % jour/mois/année
\newcommand{\dateFinStage}{01/10/2025} % jour/mois/année
\newcommand{\nomEntreprise}{École nationale des ponts et chaussées} % Nom de l'entreprise 
\newcommand{\sujetMission}{DigiEV: A digital twin for electric vehicle charging
infrastructure in the E4C ecosystem} % Sujet de mission 
\newcommand{\supervisor}{Dr.Daphne Tuncer (École des ponts)}
\newcommand{\cosupervisor}{Dr.Georgios Bouloukakis (Télécom SudParis)}
\newcommand{\attestation}{
  \vskip 0.5em
  \justifying
  Je, soussignée Daphne Tuncer, maître de stage de l'élève Isep, ZHAO Chao, atteste avoir pris connaissance du
  cahier des charges du rapport de stage Isep, avoir lu et évalué le présent rapport au regard
  du cahier des charges et des pratiques de mon employeur, et autorisée sa transmission à l'Isep.
  \vskip 0.5em
  \hfill Signature / Date :
}





\setlength{\headheight}{14.80524pt}
%%% Packages %%%
\usepackage{multirow}
\usepackage[T1]{fontenc}
\usepackage{ragged2e}
\usepackage{listings}
%%\usepackage[latin1, utf8]{inputenc}
\usepackage{lmodern} 
\usepackage[a4paper,right=2.75cm, left=2.75cm, top=3.cm, bottom=4.cm]{geometry}
\usepackage[english]{babel} % You can change to english if needed
\usepackage{afterpage}
\usepackage[colorlinks=true, urlcolor=blue, citecolor=blue, linkcolor=black]{hyperref}
% \setSingleSpace{1.1} % Space between lines of the text
\usepackage{calc, blindtext}
\usepackage[table]{xcolor}
\usepackage{nicematrix, booktabs}
\usepackage{graphicx,soul}
\usepackage{enumitem,lipsum}
\usepackage{float} % For the [H] option in tables
\usepackage{minted}
\usepackage{adjustbox}
\usepackage{mathtools,amsfonts,amsthm,amssymb,amsmath}
\usepackage{fancyvrb}  % pour afficher proprement du code LaTeX
\usepackage{array}
\usepackage{siunitx}
\sisetup{detect-all} % note: just one option still needs to be specified
\usepackage{tikz}
\definecolor{doc}{RGB}{0,60,110}
\usepackage{titletoc}
\setcounter{tocdepth}{3}
\usepackage{pdfpages}
\usepackage{titletoc}
\usepackage[figurename=Fig., tablename=Tab.]{caption}
\usepackage{subcaption}
\usepackage[acronym]{glossaries}
\makeglossaries
\usepackage{nameref}
\renewcommand{\thesection}{\arabic{section}}

%%% Colors %%%
\definecolor{CentraleBlue}{RGB}{11,15,157}   % Blue of the Centrale Med Logo
\definecolor{codebg}{rgb}{0.95,0.95,1}       % Light blue background for the code box

%%% Custom Chapter Style %%%
\makechapterstyle{HansenColor}{
  \chapterstyle{Hansen}               
  \renewcommand{\printchaptername}{}  % Remove "Chapter" word
  \renewcommand{\printchapternum}{}   % Remove chapter number
  \renewcommand{\afterchapternum}{}   % Remove spacing left by the number

  % Set the chapter title in blue (customize color as needed)
  \renewcommand*{\printchaptertitle}[1]{\chaptitlefont\color{CentraleBlue}##1} 
}

%%% Maths %%% 

% Theorems, definitions, propositions etc.
\theoremstyle{plain}
\newtheorem{theorem}{Théorème}[section]
\newtheorem{proposition}[theorem]{Proposition}
\newtheorem{lemma}[theorem]{Lemme}
\newtheorem{corollary}[theorem]{Corollaire}

\theoremstyle{definition}
\newtheorem{definition}[theorem]{Définition}
\newtheorem{remark}[theorem]{Remarque}

% You can define your own commands
\newcommand{\R}{\mathbf{R}}
\newcommand{\N}{\mathbf{N}}
\newcommand{\Q}{\mathbf{Q}}
\newcommand{\Z}{\mathbf{Z}}
\newcommand{\CC}{\mathbf{C}}
\newcommand{\E}{\mathbf{E}}
\newcommand{\D}{\ensuremath{\mathrm{d}}}
\newcommand{\Cov}{\mathrm{Cov}}
\newcommand{\Var}{\mathrm{Var}}

\input{MainLayout/PageLayout}

\input{MainLayout/ChapterStyle}

\makeatother


\chapterstyle{HansenColor}

\begin{document}

\setcounter{secnumdepth}{3}
\thispagestyle{empty}
\newcommand{\HRule}{\rule{\linewidth}{0.5mm}} % épaisseur de lignes horizontales
\setlength\fboxrule{2pt} % épaisseur de la boîte de mission
\newgeometry{top=2cm, bottom=1cm, left=2cm, right=2cm}

%----------------------------------------------------------------------------------------
%	SECTION LOGO
%----------------------------------------------------------------------------------------
\begin{center}
    \adjustbox{valign=c}{\includegraphics[width=3cm]{Images/isep.png}}
    \hspace{1.5cm}
    \adjustbox{valign=c}{\includegraphics[width=6cm]{Images/logo-enpc-ip-rvb.jpg}}
    \hspace{1cm}
    \adjustbox{valign=c}{\includegraphics[width=5cm]{Images/logo_tsp.png}}
\end{center}

%--------------------------prenomNOM--------------------------------------------------------------
%	SECTION TITRE
%----------------------------------------------------------------------------------------
\begin{center}
\definecolor{lightgray}{gray}{0.9}
\vskip 3em
\fcolorbox{black}{lightgray}{
    \begin{minipage}[c][5em][c]{0.9\textwidth}
        \centering
        \textbf{\LARGE\sujetMission}
    \end{minipage}
}
\vskip 3em
\Large\textbf{\LARGE Report of End of Studies Internship (IS.3001)}
\vskip 0.5em
presented by \textbf{\prenomNOM}
\vskip 0.5em
student id: \textbf{63315}

\end{center}

% %----------------------------------------------------------------------------------------
% %	SECTION MISSION DE STAGE
% %----------------------------------------------------------------------------------------
% \begin{center}
%     \Large
%     \definecolor{lightgray}{gray}{0.9}
%     \vskip 2em
%     \fcolorbox{black}{lightgray}{
%         \begin{minipage}[c][6em][c]{0.9\textwidth}
%             \centering
%             \textbf{\sujetMission}
%         \end{minipage}
%     }
%     \end{center}


%----------------------------------------------------------------------------------------
% SECTION TUTEUR/DIRECTEUR STAGE
%----------------------------------------------------------------------------------------
\begin{center}
\Large
\vskip 3em
Internship from \dateDebutStage \ to \dateFinStage \ at :
\vskip 0.5em
\textbf{\nomEntreprise}
\vskip 2em
\begin{minipage}[l]{0.95\textwidth}
    \begin{flushleft}
    Supervisor : \textbf{\supervisor}
    \vskip 0.5em
    Co-Supervisor : \textbf{\cosupervisor}
    \end{flushleft}
    \vskip 1em
    \HRule
\end{minipage}
\end{center}
%----------------------------------------------------------------------------------------
% SECTION BAS DE PAGE
%----------------------------------------------------------------------------------------
\begin{center}
\Large
\vskip 1em
Attestation du maître de stage : \attestation
\end{center}
\vfill % remplit le reste de la page avec des espace

\restoregeometry
\cleardoublepage
\normalsize


\newgeometry{top=3cm, bottom=1cm, left=2cm, right=6cm}  % set new margins


% --- Remerciements ---
\chapter*{Acknowledgments}

I would like to sincerely thank my supervision team, Dr.Daphne Tuncer and Dr.Georgios Bouloukakis, for their continuous guidance, insightful advice, and encouragement throughout this six-month internship. Their expertise not only supported my R\&D tasks but also inspired me to explore the world of academic research, opening a new path in my professional journey.

I am also deeply grateful to my wife, WANG Yaoyao, for her unconditional support in daily life. Her encouragement has been a vital source of strength during this period. In addition, I would like to express my heartfelt gratitude to my parents, whose love, trust, and support have always been the foundation of my personal and professional growth. All their constant encouragement has given me the confidence to pursue new challenges and persist through difficulties.

Lastly, I would like to acknowledge E4C for providing the resources and support that made this internship possible.


% \bigskip
% % \begin{center}
% %     \Huge * * *
% % \end{center}

\newpage


% --- Abstract ---

\chapter*{Abstract}
% ENGLISH ABSTRACT

The success of electric mobility requires not only advanced electric vehicles but also an integrated infrastructure that combines electricity provision, information distribution, and intelligent system control. Digital twin technology has emerged as a powerful approach to manage complex infrastructure operations and evaluate deployment strategies for electric vehicle charging networks.

This report describes the work conducted during my six-month internship at Laboratoire Ville Mobilité Transport (LVMT), focusing on the design and development of a digital twin system to simulate the energy impact of electric vehicle charging infrastructure. The project employed FIWARE technology as the core framework, utilizing Python for system development and data processing to model the complex interactions between energy consumption and user behavior in charging station.

The developed digital twin system successfully established a comprehensive platform for simulating charging station operations, built upon NGSI-LD standard compliance to ensure semantic interoperability. The system utilizes MongoDB and CrateDB as backend databases for efficient data storage and time-series analysis, while providing RESTful API interfaces for seamless system integration and external interactions. This platform enables real-time monitoring and analysis of charging infrastructure performance, providing valuable insights for optimizing charging station deployment, predicting energy demand patterns, and improving overall system efficiency. The work contributes to enhanced decision-making capabilities for electric vehicle charging infrastructure planning and management through standardized data models and accessible programming interfaces.

This internship significantly enhanced my technical expertise. Additionally, I strengthened my research skills through extensive literature review and analysis of current challenges in sustainable transportation infrastructure. The experience provided valuable insights into the interdisciplinary nature of smart city technologies and the critical role of digital twins in managing complex urban systems.

\bigskip\bigskip

\textbf{\color{CentraleBlue}Keywords} Digital Twin, Simulator, Semantic Model, Electric Vehicle, Charging Infrastructure, FIWARE

\newpage

\restoregeometry 

\tableofcontents

%%% Chapters %%%
\chapter{Introduction}\label{chap1}

The global transition toward electric mobility is regarded as a cornerstone of 
efforts to decarbonize the transport sector and meet international climate 
targets. According to the International Energy Agency, electric car sales 
reached nearly 14 million in 2023, and are projected to exceed 17 million 
in 2024, representing more than one-fifth of global car sales~\cite{IEA2024}. 
However, this rapid diffusion of electric vehicles (EVs) poses systemic 
challenges that extend beyond vehicle manufacturing. The widespread adoption 
of EVs crucially depends on the development of robust charging infrastructure 
and the integration of energy systems capable of supporting increasing loads 
while ensuring reliability and user acceptance~\cite{Metais2022}. 

Charging infrastructure represents both a key enabler and a potential bottleneck 
in the electrification of mobility. Without adequate charging coverage, 
prospective EV users may face range anxiety, undermining confidence in the 
technology and slowing adoption~\cite{Metais2022}. At the same time, 
deploying charging networks entails high capital costs and introduces 
technical challenges related to grid integration, station placement, and 
operational optimization. France, for instance, has set ambitious goals of 
400,000 publicly accessible charging points by 2030, yet as of 2023 fewer 
than 100,000 had been deployed, revealing the scale of investment and 
coordination required~\cite{Autorite2023}. This “chicken-and-egg” 
problem—insufficient infrastructure limiting EV adoption, while weak demand 
discourages infrastructure investment—highlights the importance of 
strategically planned deployment~\cite{Bernard2021}.

To address these challenges, advanced modeling and simulation tools have been 
proposed to guide infrastructure planning and management. Digital twin 
technology, in particular, offers a powerful approach to replicate complex 
physical systems in a virtual environment, enabling real-time monitoring, 
scenario evaluation, and predictive analytics. Recent studies highlight the 
potential of digital twins to support the integration of charging infrastructure 
with smart grids, optimize location and capacity decisions, and enhance 
decision-making under uncertainty~\cite{Charette2023, Metais2022}. 


This report builds upon this context by presenting the work conducted during 
a six-month internship at the Laboratoire Ville Mobilité Transport (LVMT), 
École nationale des ponts et chaussées (ENPC). The project focused on the 
design and development of a digital twin system for simulating the energy 
impact of EV charging infrastructure, employing the FIWARE framework and 
NGSI-LD standards to ensure interoperability. By coupling backend databases 
(MongoDB, CrateDB) with real-time APIs, the system provides a scalable 
platform for monitoring charging operations and assessing deployment 
strategies. The work contributes to ongoing research in sustainable mobility 
by demonstrating how digital twins can support the optimization of EV 
charging infrastructure within smart city ecosystems.

This report is organized into seven main chapters, each addressing specific aspects 
of the internship work and its broader context:

Chapter 2 (\textbf{Context}) presents the institutional framework of the internship, 
including the presentation of ENPC, LVMT, E4C, and Saclay campus, along with supervisor 
introductions and the background context of energy and climate research, EV charging 
infrastructure, digital twin technologies, and semantic data modeling.

Chapter 3 (\textbf{Literature Review}) examines existing research and technologies, 
including FIWARE, BRICK, Web of Things (WoT), and related digital twin implementations. 
This chapter identifies current challenges and positions our contribution within the 
existing body of knowledge.

Chapter 4 (\textbf{Methods}) details the system architecture, semantic data modeling 
approach using NGSI-LD, backend database implementation, API development, and frontend 
visualization components. It concludes with an analysis of contributions, difficulties 
encountered, and deviations from initial specifications.

Chapter 5 (\textbf{Case Studies and Experiments}) presents the definition and 
initialization of three experimental scenarios, results obtained using both real 
and synthetic data, and validation of the digital twin system performance.

Chapter 6 (\textbf{Management Analysis}) addresses social and environmental 
responsibility aspects of the project and discusses innovation management and digital 
tools for decision-making in resear



\chapter{Quelques commandes utiles}\label{chap2}

\section{Inclure des images}

\begin{figure}[ht!]
    \centering
    \includegraphics[width=0.7\textwidth]{example-image-a}
    \caption{\textbf{Titre.} \lipsum[1][1-5]}
    \label{fig:image1}
\end{figure}

\begin{figure}[ht!]
    \centering
    \includegraphics[width=0.35\textwidth]{example-image-a}
    \includegraphics[width=0.35\textwidth]{example-image-b}
    \caption{\textbf{Titre.} \lipsum[1][1-5]}
    \label{fig:image2}
\end{figure}



\section{Listes et tableaux}

\textbf{Insérer une liste~:}
\begin{itemize}
    \item Premier niveau
    \begin{itemize}
        \item[(i)] Deuxième niveau
        \item[(ii)] Un autre élément au deuxième niveau
    \end{itemize}
    \item Un autre élément au premier niveau
        \begin{itemize}
        \item[(a)] Deuxième niveau
        \item[(b)] Un autre élément au deuxième niveau
    \end{itemize}
\end{itemize}
\bigskip

\textbf{Insérer un tableau simple~:} 
\begin{table}[H]
    \centering
    \begin{tabular}{|c|c|c|c|c|c|c|c|c|c|}
        \hline
        A & B & C & D & E & F & G & H & I & \dots \\
        \hline
        1 & 2 & 3 & 4 & 5 & 6 & 7 & 8 & 9 & \dots\\
        10 & 11 & 12 & 13 & 14 & 15 & 16 & 17 & 18 & \dots \\
        \hline
    \end{tabular}
    \caption{\textbf{Titre.} \lipsum[1][1-3]}
    \label{tab:table-label}
\end{table}

\textbf{Autre style de tableau~:} 
\begin{table}[H]
    \centering
    \begin{tabular}{c c c c c}
        \hline
        \textbf{Col1} & \textbf{Col2} & \textbf{Col3} & \textbf{Col4} & \textbf{Col5} \\
        \hline
        1 & 2 & 3 & 4 & 5 \\
        1 & 2 & 3 & 4 & 5 \\
        \hline
    \end{tabular}
    \caption{\textbf{Titre.} \lipsum[1][1-3]}
    \label{tab:table-label}
\end{table}


\section{Insertion de code Python}
\textbf{Insérer du code en \textit{inline}~:} \texttt{print("Hello, World!")}.\\


\textbf{Insérer du code en \textit{inline} avec coloration syntaxique~:} \mintinline{python}{print("Hello, World!")}.\\

\textbf{Insérer du code en bloc avec coloration syntaxique~:} 
\begin{minted}[bgcolor=codebg,fontsize=\small,frame=lines,linenos]{python}
def factorial(n):
    if n == 0:
        return 1
    else:
        return n * factorial(n - 1)

# Example usage
print(factorial(5))  # Output: 120
\end{minted}



\section{Expressions mathématiques}
\subsection{Théorèmes, propositions, définitions, lemmes, demonstrations...}

\begin{theorem}
    \lipsum[1][1-4]
\end{theorem}

\begin{proof}
    \lipsum[1][1-4]
\end{proof}

\begin{proposition}
    \lipsum[1][1-4]
\end{proposition}

\begin{definition}
    \lipsum[1][1-4]
\end{definition}

\begin{remark}
    \lipsum[1][1-4]
\end{remark}

\begin{lemma}
    \lipsum[1][1-4]
\end{lemma}


\subsection{Equations et calculs sur plusieurs lignes}

\subsubsection*{Une simple équation}
\begin{equation}
    e = mc^2
\end{equation}

\subsubsection*{Une équation sur plusieurs lignes}
\begin{equation}
    \begin{split}
        \mathbb{E}(aX + Y) &= \mathbb{E}(aX) + Y\\
                           &= a\mathbb{E}(X) + Y
    \end{split}
\end{equation}

\subsubsection*{Une équation avec plusieurs cas}
\begin{equation}
    u_n =
    \begin{cases}
        1 \text{ if } n\equiv0 \mod 2\\
        0 \text{ if } n\equiv1 \mod 2 \\
    \end{cases}
\end{equation}

\subsubsection*{Insérer une série de calculs}
\begin{align*}
    x &= 0.999\ldots \\
    10x &= 9.999\ldots \\
    10x - x &= 9.999\ldots - 0.999\ldots \\
    9x &= 9 \\
    x &= 1 \\
    0.999\ldots &= 1
\end{align*}

\subsection{Utilisation du glossaire}

\newglossaryentry{esperance}{
    name=espérance,
    description={Valeur moyenne théorique d'une variable aléatoire}
}


Référence au glossaire : \gls{esperance}.  







%%% Glossaire %%%
\printglossaries


%%% Appendix %%%
\appendix

\chapter{Annexe}
\subsection*{\Large Annexe 1 - List of Interviewees (Chapter 6)}

\begin{itemize}
    \item \textbf{Dr.Daphne Tuncer:} Chercheure ENPC
    \item \textbf{Dr.Georgios Bouloukakis:} Assistant Professor at University of Patras
\end{itemize}


\includepdf[pages=1,scale=0.8,pagecommand={
    \begin{minipage}{\textwidth}
    \centering
    \vspace{0cm}
    {\Large \textbf{Annexe 2 - Internship Advertisement}}
    \raggedright
    \end{minipage}
}]{Images/electricMobilityDigitalTwin_projectDescription_2025.pdf}
\includepdf[pages=2-,scale=0.8]{Images/electricMobilityDigitalTwin_projectDescription_2025.pdf}


\includepdf[pages=1,scale=0.8,pagecommand={
    \begin{minipage}{\textwidth}
    \centering
    \vspace{0cm}
    {\Large \textbf{Annexe 3 - CV}}
    \raggedright
    \end{minipage}
    }]{Images/cv.pdf}
    \newpage


\includepdf[pages=1,scale=0.8,pagecommand={
    \begin{minipage}{\textwidth}
    \centering
    \vspace{0cm}
    {\Large \textbf{Annexe 4 - Motivation Letter}}
    \raggedright
    \end{minipage}
}]{Images/Motivation Letter.pdf}
\includepdf[pages=2-,scale=0.8]{Images/Motivation Letter.pdf}




\urlstyle{same}


%%% Bibliography %%%
\bibliographystyle{plain} 
\bibliography{References}

\end{document}

